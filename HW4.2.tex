\documentclass[12pt]{article}   % The [12pt] is optional for senior eyes; the default is 10 pt

\setlength{\parindent}{0pt}
\pdfpagewidth 8.5in
\pdfpageheight 11in

\usepackage{amsmath}  % It's generally a good idea to include this if you're going to use any "unusual" symbols. 
\usepackage{amssymb}  % More out of the way math symbols such as \nmid for does not divide
\usepackage{amsfonts}  % I am using this to get the "Blackboard Bold" symbols. 

\usepackage{enumerate}  % The standard enumerate environment numbers the items consecutively. This package enables me to use letters instead. It's optional.
\usepackage{amsthm}
\usepackage{enumitem}


\usepackage{mdwlist}
\usepackage[margin=1.2in]{geometry}

% The following are definitions for shortcuts for pieces of notation that I use a lot.
\newcommand{\N}{\mathbb{N}} % the natural numbers
\newcommand{\Z}{\mathbb{Z}} % the integers
\newcommand{\C}{\mathbb{C}} % the complex numbers 
\newcommand{\R}{\mathbb{R}} % the real numbers
\newcommand{\Q}{\mathbb{Q}} % the rational numbers
\newcommand{\F}{\mathbb{F}} % the Field
\renewcommand{\a}{\alpha}
\renewcommand{\b}{\beta}
\newcommand{\lgr}{\lambda}
\newcommand{\from}{\!\!:} % squeeze the colon closer to the name of the domain of a function
\newcommand{\iso}{\cong} % isomorphism
\newcommand{\rmth}{^{\text{th} } } % roman th as superscript
\newcommand{\inv}{^{-1}} % inverse (exponent -1)
\newcommand{\units}{^{\times}} % groups of units as in E^x

% operator names (set in Roman to contrast with math letters 
\newcommand{\Aut}{\operatorname{Aut}}
\newcommand{\sgn}{\operatorname{sgn}}
\newcommand{\Sym}{\operatorname{Symm}}
\newcommand{\lcm}{\operatorname{lcm}}
\newcommand{\rad}{\operatorname{rad}}
\newcommand{\Gal}{\operatorname{Gal}}

% lower case boldface letters
\newcommand{\uu}{\mathbf u}
\newcommand{\vv}{\mathbf v}
\newcommand{\ww}{\mathbf w}
\newcommand{\ii}{\mathbf i}
\newcommand{\jj}{\mathbf j}
\newcommand{\kk}{\mathbf k}


\newtheorem{problem}{Problem}
\newtheorem{corollary}{Corollary}[theorem]
\newtheorem{lemma}[theorem]{Lemma}

\newcommand{\th}{^{\text{th} } } % roman th as superscript
\newcommand{\st}{^{\text{st} } } % roman st as superscript
\newcommand{\nd}{^{\text{nd} } } % roman nd as superscript

\newenvironment{amatrix}[1]{%
  \left[\begin{array}{@{}*{#1}{c}|c@{}}
}{%
  \end{array}\right]
}


\title{MATH544 \\ Dr. Miller \\ HW 4}
\author{Justin Baum}

\begin{document}

\maketitle

\begin{problem}[2.2.2]
Give an explicit isomorphism between $\R^2$ and the set of solutions of the linear system over($\R$)
\[\begin{array}{ccc}
     w-x+3z&=&0  \\
     w-x+y+5z&=&0 \\
     2w-2x-y+4z&=&0
\end{array}\]
\end{problem}
\begin{proof}
The set of solutions(after throwing this into an augmented matrix and using Gaussian row elimination) is, 
\[\left \{\begin{bmatrix}s-3t\\-2s\\s\\t\end{bmatrix}\ \Bigg |\  s,t\in \R\right \}\] 
We want a transformation that hits all solutions, and is bijective. Let the transformation = $A$.
\[A\begin{bmatrix}x\\y \end{bmatrix}=\begin{bmatrix}x-3y\\-2y\\x\\y\end{bmatrix}\]
The following is an isomorphism from $\R^2$ to the solutions.
\[\begin{bmatrix}1 &-3 \\ 0 & -2 \\ 1 & 0 \\ 0 & 1\end{bmatrix}\]
\end{proof}
\begin{problem}[2.2.6]
Find the matrix of the orthogonal projection onto the line $y=x$ in $\R^2$.
\end{problem}
\begin{proof}
The orthogonal projection transformation should look like,
\[T\left (\begin{bmatrix}x\\y\end{bmatrix}\right)=\begin{bmatrix}y\\x\end{bmatrix}\]
We can solve this as,
\[\begin{bmatrix}a& b \\ c & d\end{bmatrix}\begin{bmatrix}x\\y\end{bmatrix}=\begin{bmatrix}y \\x\end{bmatrix}\]
When expanded, and solved we get, $\begin{bmatrix}0 & 1\\1 & 0\end{bmatrix}$.
\end{proof}
\begin{problem}[2.2.8]
Show that if 0 is an eigenvalue of $T\in L(V)$, then $T$ is not injective and therefore not invertible.
\end{problem}
\begin{proof}
If 0 is an eigenvalue of $T$, then for some $\vec{v}\in V$, and $\vec{v}\ne \vec{0}$(definition of eigenvector), $T(\vec{v})=0\vec{v}$. This is equal to $\vec{0}$. We know that because $T\in\mathcal{L}(V)$, that $T(\vec{0})=\vec{0}$, thus it is not injective. In essence, $\vec{v}\ne \vec{0}$, and $T(\vec{v})=T(\vec{0})$.
\end{proof}
\begin{problem}[G]
Let $V=\F^{\infty}$ be the infinite sequence space. Define the shift left operator on V by, $L(a_0,a_1,a_2,\dots)=(a_1, a_2, a_3,\dots)$ and the shift right operator as $R(a_0,a_1,a_2,\dots)=(0,a_0,a_1,\dots)$.
\end{problem}
\begin{problem}[G.ii]
Is $L$ injective? Surjective? If yes prove it, if no, why not?
\end{problem}
\begin{proof}
$L$ is not injective, because say we have $A:=\begin{bmatrix}1\\2\\2\\ \vdots\\2\end{bmatrix}$ and $B:=\begin{bmatrix}2\\2\\2\\ \vdots\\2\end{bmatrix}$. Both of these vectors are distinct, but $L(A)=L(B)$.\\
$L$ is surjective, for all $\vec{v}=\begin{bmatrix}a_0\\a_1\\a_2\\\vdots\end{bmatrix}\in \F^{\infty}$, so $T\left(\begin{bmatrix}x\\a_0\\a_1\\\vdots\end{bmatrix}\right ) =\begin{bmatrix}a_0\\a_1\\a_2\\\vdots\end{bmatrix}= \vec{v}$ for all $x\in\F$.
\end{proof}
\begin{problem}[G.iii]
Is $R$ injective? Surjective? If yes prove it, if no, why not?
\end{problem}

\begin{proof}
$R$ is injective, suppose $T\left (\begin{bmatrix}a_0\\a_1\\a_2\\\vdots\end{bmatrix} \right )=T\left ( \begin{bmatrix}b_0\\b_1\\b_2\\\vdots\end{bmatrix}\right )$. Then $\begin{bmatrix}0\\a_0\\a_1\\\vdots\end{bmatrix}=\begin{bmatrix}
0\\b_0\\b_1\\\vdots
\end{bmatrix}$. And so, $a_0=b_0$, $a_1=b_1$, $a_2=b_2$, $\dots$, $a_n=b_n$, thus $\begin{bmatrix}
a_0\\a_1\\a_2\\\vdots
\end{bmatrix}=\begin{bmatrix}
b_0\\b_1\\b_2\\\vdots
\end{bmatrix}$\\
$R$ is not surjective, because $\begin{bmatrix}
1\\\vdots
\end{bmatrix}\in V^{\infty}$ but there does not exist $\vec{v}$, such that $T(\vec{v})=\begin{bmatrix}
1\\\vdots
\end{bmatrix}$, because all elements in the image have the first row as zero.
\end{proof}
\begin{problem}[G.iv]
Are the compositions, $LR$ and $RL$ the same? If yes prove it; if no, why not?
\end{problem}
\begin{proof}
$LR = L\circ R$, so a transformation of such, \[L\left(R\left(
\begin{bmatrix}
a_0\\a_1\\a_2\\\vdots
\end{bmatrix}
\right)\right)=L\left(
\begin{bmatrix}
0\\a_0\\a_1\\\vdots
\end{bmatrix}
\right)=
\begin{bmatrix}
a_0\\a_1\\a_2\\\vdots
\end{bmatrix}
\]
And RL = $R\circ L$, so a transformation of such,
\[R\left(L\left(
\begin{bmatrix}
a_0\\a_1\\a_2\\\vdots
\end{bmatrix}
\right)\right)=
R\left(
\begin{bmatrix}
a_1\\a_2\\a_3\\\vdots
\end{bmatrix}
\right)=\begin{bmatrix}
0\\a_1\\a_2\\\vdots
\end{bmatrix}\]
And so these transformations are not the same.
\end{proof}
\begin{problem}[I]
Let $V = C^{\infty}(R) = \{f: R \to R\ \vert\  \text{f and all of its derivatives exist and are continuous}\}$.
This is a vector space over R
\end{problem}
\begin{problem}[I.i]
Define $D: V \to V by D(f) = f'$. Convince yourself that D is linear. Which
functions are in ker(D)?
\end{problem}
\begin{proof}
Let $z(x)=0$. This is $\vec{0}$ in this space.\\
By definition, $ker(D)=\{f\ |\ f'=z\}$. By the fundamental theorem of calculus, $ker(D)=\{f\ |\ \forall x\in\R; f(x)=0x+C\}$.
\end{proof}
\begin{problem}[I.ii]
Let $g(x)=e^{-3x}$\\
By definition for an eigenvector, the eigenvectors for which the eigenvalue, $-3$, would solve the equation.
\[D(f)=f'=-3f\]
The eigenvector $g$, works in this equation because,
\[D(g)=g'=-3\cdot e^{-3x}=-3g\]
There exists a function $f$ for every real number, $\lambda$, such that $\lambda$ is the eigenvalue because by differentiation in calculus, let $f=e^{\lambda x}$ then $\frac{df}{dx}=D(f)=\lambda f$
\end{problem}
\begin{problem}[I.iii]
The composition $D^2 = D\circ D: V \to V$ is given by $D^2(f) = f''$. Show that $-4$ is an
eigenvalue for $D^2$. Is every negative real number an eigenvalue for $D^2$?
\end{problem}
\begin{proof}
Every negative real number is an eigenvalue, because for all $\lambda \in \R$ and $\lambda < 0$, let $\beta = -\lambda$, we can construct a function, $f$ that $D^2(f)=\lambda f=-\beta f$. Let $f=e^{\sqrt{\beta}i}$, then $D(D(f))=D(\sqrt{\beta}if)=i^2 \beta f=-\beta f=\lambda f$, by differentiation rules of calculus.\\
We can now find a function that has an eigenvalue of $-4$, let $f=e^{\sqrt{4}i}$, then $D^2(f)=-4f$.
\end{proof}
\end{document}
