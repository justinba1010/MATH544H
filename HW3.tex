\documentclass[12pt]{article}   % The [12pt] is optional for senior eyes; the default is 10 pt

\setlength{\parindent}{0pt}
\pdfpagewidth 8.5in
\pdfpageheight 11in

\usepackage{amsmath}  % It's generally a good idea to include this if you're going to use any "unusual" symbols. 
\usepackage{amssymb}  % More out of the way math symbols such as \nmid for does not divide
\usepackage{amsfonts}  % I am using this to get the "Blackboard Bold" symbols. 

\usepackage{enumerate}  % The standard enumerate environment numbers the items consecutively. This package enables me to use letters instead. It's optional.
\usepackage{amsthm}

\usepackage{mdwlist}
\usepackage[margin=1.2in]{geometry}

% The following are definitions for shortcuts for pieces of notation that I use a lot.
\newcommand{\N}{\mathbb{N}} % the natural numbers
\newcommand{\Z}{\mathbb{Z}} % the integers
\newcommand{\C}{\mathbb{C}} % the complex numbers 
\newcommand{\R}{\mathbb{R}} % the real numbers
\newcommand{\Q}{\mathbb{Q}} % the rational numbers
\newcommand{\F}{\mathbb{F}} % the Field
\renewcommand{\a}{\alpha}
\renewcommand{\b}{\beta}
\newcommand{\lgr}{\lambda}
\newcommand{\from}{\!\!:} % squeeze the colon closer to the name of the domain of a function
\newcommand{\iso}{\cong} % isomorphism
\newcommand{\rmth}{^{\text{th} } } % roman th as superscript
\newcommand{\inv}{^{-1}} % inverse (exponent -1)
\newcommand{\units}{^{\times}} % groups of units as in E^x

% operator names (set in Roman to contrast with math letters 
\newcommand{\Aut}{\operatorname{Aut}}
\newcommand{\sgn}{\operatorname{sgn}}
\newcommand{\Sym}{\operatorname{Symm}}
\newcommand{\lcm}{\operatorname{lcm}}
\newcommand{\rad}{\operatorname{rad}}
\newcommand{\Gal}{\operatorname{Gal}}

% lower case boldface letters
\newcommand{\uu}{\mathbf u}
\newcommand{\vv}{\mathbf v}
\newcommand{\ww}{\mathbf w}
\newcommand{\ii}{\mathbf i}
\newcommand{\jj}{\mathbf j}
\newcommand{\kk}{\mathbf k}


\newtheorem{problem}{Problem}
\newtheorem{corollary}{Corollary}[theorem]
\newtheorem{lemma}[theorem]{Lemma}

\newcommand{\th}{^{\text{th} } } % roman th as superscript
\newcommand{\st}{^{\text{st} } } % roman st as superscript
\newcommand{\nd}{^{\text{nd} } } % roman nd as superscript



\title{MATH544 \\ Dr. Miller \\ HW 3}
\author{Justin Baum}
\date{February 4, 2019}


\begin{document}

\maketitle

\begin{problem}[1.4.7]
Show that $ad-bc\neq 0$, then the linear system
\[ax+by=e\]
\[cx+dy=f\]
over a field $\F$ has a unique solution.
\end{problem}
\begin{proof}
Multiply the first equation by $c$ and the second by $a$.
\[acx+bcy=ce\]
\[acx+ady=af\]
\[ady-bcy=af-ce\]
\[(ad-bc)y=af-ce\]
\[y=\frac{af-ce}{ad-bc}\]
We can then extend this argument again with $d$ and $b$.
\[adx+bdy=de\]
\[bcx+bdy=bf\]
\[adx-bcx=de-bf\]
\[x=\frac{de-bf}{ad-bc}\]
Thus there exists a unique solution for a 2x2 linear system as long as $ad-bc\neq0$.
\end{proof}
\begin{problem}[Supp A]
\[a_{11}x_1+\dots+a_{1n}x_n=\lambda x_1\]
\[\vdots\]
\[a_{n1}x_1+\dots+a_{nn}x_n=\lambda x_n\]
This can be written as,
\[(a_{11}-\lambda)x_1+a_{21}x_2+\dots+a_{1n}x_n=0\]
\[\vdots\]
\[a_{b1}x_1+\dots+(a_{bb}-\lambda)x_b+\dots+a_{bn}x_n=0\]
\[\vdots\]
\[a_{n1}x_1+a_{n2}x_2+\dots+(a_{nn}-\lambda)x_n=0\]
If $x_1=c_1,\dots,x_n=c_n$ and $x_1=d_1,\dots,x_n=d_n$ are solutions, is $x_1=rc_n,\dots,x_n=rc_n$ a solution?
\[(a_{11}-\lambda)rc_1+a_{21}rc_2+\dots+a_{1n}rc_n=r((a_{11}-\lambda)c_1+a_{21}c_2+\dots+a_{1n}c_n)=r\cdot 0=0\]
This argument can be extended to each equation in the system. Thus this works for all $r\in \R$. \\
Assume $x_1=c_1,\dots,x_n=c_n$ and $x_1=d_1,\dots,x_n=d_n$ are solutions.\\
Let $x_1=c_1+d_1,\dots, x_n=c_n+d_n$.
\[(a_{11}-\lambda)(c_1+d_1)+a_{21}(c_1+d_1)+\dots+a_{1n}(c_n+d_n)\]
\[\vdots\]
\[a_{b1}(c_1+d_1)+\dots+(a_{bb}-\lambda)(c_b+d_b)+\dots+a_{bn}(c_n+d_n)\]
\[\vdots\]
\[a_{n1}(c_1+d_1)+a_{n2}(c_2+d_2)+\dots+(a_{nn}-\lambda)(c_n+d_n)\]
This can be rewritten as
\[\left ((a_11-\lambda)c_1+\dots+a_{1n}c_n\right )+\left ((a_11-\lambda)c_1+\dots+a_{1n}c_n)\right )=2\cdot0\]
Which is the addition of two solutions where all the equations are homogeneous. This argument can be extended to each equation.
\end{problem}

\begin{problem}
Find the values of $\lambda$ does the equations have a non trivial solution. In each case give a sample of a nontrivial solution.
\[2x+y=\lambda x\]
\[x+2y=\lambda y\]
\end{problem}
\begin{proof}
\[(2-\lambda)x+y=0\]
\[x+(2-\lambda)y=0\]
Thus $y=\frac{x}{\lambda-2}$ and $y=(\lambda-2)x$.
We can say then,
\[\frac{x}{\lambda-2}=(\lambda-2)x\]
\[x=(\lambda^2-4\lambda+4)x\]
\[-x+(\lambda^2-4\lambda+4)x=0=(\lambda^2-4\lambda+3)x\]
The trivial solurion is when $x=0$, so instead we will try when $\lambda^2-4\lambda+3=0$, which is true when $\lambda=1$ and $\lambda=3$.\\
When $\lambda=1$, the equations become, 
\[x+y=0\]
\[x+y=0\]
Which is solved when for all $y=-x$ for all $x      \in\R$, and when $\lambda=3$
\[-x+y=0\]
\[x-y=0\]
Which is solved when $y=x$ for all $x\in\R$.
\end{proof}

\end{document}