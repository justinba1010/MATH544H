\documentclass[12pt]{article}   % The [12pt] is optional for senior eyes; the default is 10 pt

\setlength{\parindent}{0pt}
\pdfpagewidth 8.5in
\pdfpageheight 11in

\usepackage{amsmath}  % It's generally a good idea to include this if you're going to use any "unusual" symbols. 
\usepackage{amssymb}  % More out of the way math symbols such as \nmid for does not divide
\usepackage{amsfonts}  % I am using this to get the "Blackboard Bold" symbols. 

\usepackage{enumerate}  % The standard enumerate environment numbers the items consecutively. This package enables me to use letters instead. It's optional.
\usepackage{enumitem}

\usepackage{amsthm}

\usepackage{mdwlist}
\usepackage[margin=1.2in]{geometry}

% The following are definitions for shortcuts for pieces of notation that I use a lot.
\newcommand{\N}{\mathbb{N}} % the natural numbers
\newcommand{\Z}{\mathbb{Z}} % the integers
\newcommand{\C}{\mathbb{C}} % the complex numbers 
\newcommand{\R}{\mathbb{R}} % the real numbers
\newcommand{\Q}{\mathbb{Q}} % the rational numbers
\newcommand{\F}{\mathbb{F}} % the Field
\renewcommand{\a}{\alpha}
\renewcommand{\b}{\beta}
\newcommand{\lgr}{\lambda}
\newcommand{\from}{\!\!:} % squeeze the colon closer to the name of the domain of a function
\newcommand{\iso}{\cong} % isomorphism
\newcommand{\rmth}{^{\text{th} } } % roman th as superscript
\newcommand{\inv}{^{-1}} % inverse (exponent -1)
\newcommand{\units}{^{\times}} % groups of units as in E^x
\newcommand{\vu}{u}

% operator names (set in Roman to contrast with math letters 
\newcommand{\Aut}{\operatorname{Aut}}
\newcommand{\sgn}{\operatorname{sgn}}
\newcommand{\Sym}{\operatorname{Symm}}
\newcommand{\lcm}{\operatorname{lcm}}
\newcommand{\rad}{\operatorname{rad}}
\newcommand{\Gal}{\operatorname{Gal}}

% lower case boldface letters
\newcommand{\uu}{\mathbf u}
\newcommand{\vv}{\mathbf v}
\newcommand{\ww}{\mathbf w}
\newcommand{\ii}{\mathbf i}
\newcommand{\jj}{\mathbf j}
\newcommand{\kk}{\mathbf k}


\newtheorem{problem}{Problem}
\newtheorem{corollary}{Corollary}[theorem]
\newtheorem{lemma}[theorem]{Lemma}

\newcommand{\th}{^{\text{th} } } % roman th as superscript
\newcommand{\st}{^{\text{st} } } % roman st as superscript
\newcommand{\nd}{^{\text{nd} } } % roman nd as superscript



\title{MATH544 \\ Dr. Miller \\ HW 3}
\author{Justin Baum}

\begin{document}

\maketitle

\begin{problem}[1.4.2]
Let $\F = \{a+bi|\ a,b\in\Q\}$, show that $\F$ is a field.
\end{problem}
\begin{proof}
In order for $\F$ to be a field it would need to follow that,\\
Note it is redundant that $a,b\in\Z \rightarrow$ ($a+b\in\Z$ and $a\cdot b\in\Z$), and $\frac{a}{b}\in\Q$ if $a,b\in\Z$, so it will be omitted from the rest of the proof.
\begin{enumerate}
    \item $\F$ is closed and commutative under addition.\\
    Let $u, v \in \F$ then ($a,b,c,d,e,f,g,h\in\Z$)
    \[u+v=\frac{a}{b}+\frac{c}{d}i+\frac{e}{f}+\frac{g}{h}i=\frac{af+be}{bf}+\frac{ch+fg}{dh}i=v+u\]
    \item $\F$ is associative under addition.\\
    Let $u, v,w \in \F$ then ($a,b,c,d,e,f,g,h,j,k,l,m\in\Z$)
    \[(u+v)+v=(\frac{a}{b}+\frac{c}{d}i+\frac{e}{f}+\frac{g}{h}i)+\frac{j}{k}+\frac{l}{m}i\]
    \[=(\frac{afk+bek+bfj}{bfk}+\frac{chm+fgm+dhl}{dhm}i)=u+(v+w)\]
    \item $\F$ is closed and commutative under multiplication.\\
    Let $u, v \in \F$ then ($a,b,c,d,e,f,g,h\in\Z$)
    \[u+v=(\frac{a}{b}+\frac{c}{d}i)\cdot(\frac{e}{f}+\frac{g}{h}i)=\frac{ae}{bf}-\frac{cg}{dh}+(\frac{ce}{df}+\frac{ag}{bh})i\]
    \[=\frac{adeh-bcgf}{bdfh}+\frac{bceh+adgf}{bdfh}i=v+u\]
    \item $\F$ is associative under multiplication.\\
    Let $u, v,v \in \F$ then ($a,b,c,d,e,f,g,h,j,k,l,m\in\Z$)
    \[(u\cdot v)\cdot w=((\frac{a}{b}+\frac{c}{d}i)\cdot(\frac{e}{f}+\frac{g}{h}i))\cdot(\frac{j}{k}+\frac{l}{m}i)\]
    \[(\frac{adeh-bcfg}{bdfh}+\frac{bceh+adgf}{bhdf}i)\cdot(\frac{j}{k}+\frac{l}{m}i)\]
    \[=\frac{mjadeh-mjbcfg-lbcehk-ladgf}{bhdfmk}+\frac{mjbceh+mjadgf+kladeh-klbcfg}{bhdfkm}i\]\[=u\cdot(v\cdot w)\]
    \item $ 0$ is an additive identity.\\
    Let $u\in \F$.
    \[u+ 0=(\frac{a}{b}+\frac{c}{d}i)+(0+0i)=\frac{a}{b}+\frac{c}{d}i=u\]
    \[ 0=0+0i\]
    \item Every element has an additive inverse. $\frac{a}{b}\in\Q\rightarrow -1\frac{a}{b}\in\Q$.\\
    Let $u\in\F$, $a,b,c,d\in\Z$.
    \[\vu + (-\vu) = (\frac{a}{b}+\frac{c}{d}i)+(-\frac{a}{b}-\frac{c}{d}i)=0+0i= 0\]
    \item $1$ is a multiplicative identity. Trivial. $ 1=1+0i$
    \item Every nonzero element has a multiplicative inverse.\\
    Let $\vu \in \F$, $a,b,c,d\in\Z$.
    \[\vu\cdot\vu^{-1} = 1\]
    \[(\frac{a}{b}+\frac{c}{d}i)\vu^{-1}=1\]
    \[\vu^{-1}=\frac{1}{(\frac{a}{b}+\frac{c}{d}i)}\]
    \[\frac{1}{(\frac{a}{b}+\frac{c}{d}i)}=\frac{1}{(\frac{a}{b}+\frac{c}{d}i)}\cdot\frac{\frac{a}{b}-\frac{c}{d}i}{\frac{a}{b}-\frac{c}{d}i}=\frac{\frac{a}{b}-\frac{c}{d}i}{\frac{a^2}{b^2}+\frac{c^2}{d^2}}=\frac{\frac{a}{b}-\frac{c}{d}i}{\frac{d^2a^2+b^2c^2}{d^2b^2}}=\frac{ad^2b+cdb^2i}{d^2a^2+b^2c^2}\]
    \[\vu^{-1}=\frac{ad^2b+cdb^2i}{d^2a^2+b^2c^2}\]
\end{enumerate}
\end{proof}
\begin{problem}[1.4.3d]
Explain why $\F=\{\frac{m}{2^n}|\ m,n\in\Z\}$ is not a field. 
\end{problem}
\begin{proof}
Fails multiplicative inverse, $\vu=\frac{3}{2}\in \F$, $\vu^{-1}=\frac{2}{3}\in\Q$ but $log_2(3)\not\in \Z\rightarrow \vu^{-1}\not\in\F$.
\end{proof}
\begin{problem}[1.5.2]
Show that the set of solutions of a nonhomogeneous m x n linear system is never a subspace of $\F^n$.
\end{problem}
\begin{proof}
\[a_{11}x_1+\dots+a_{1n}x_n=b_1\]
\[\vdots\ \ \ \ \ \ddots\ \ \ \ \ \ \vdots\ \ \ \ \ \ \ \ \vdots\]
\[a_{m1}x_1+\dots+a_{mn}x_n=b_m\]
For this set of equations at least one of the equations has a $b\neq 0$, we'll call this row, row j, where $b_j \neq 0$.
\[a_{j1}x_1+\dots + a_{jn}= b_j\ |\ b_j \neq 0\]
Thus this set of solutions fails closure over additions. Say $x_1=c_!,\dots x_n=c_n$ and $x_1=d_1,\dots,x_n=d_n$ are solutions. $x_1=c_1+d_1,\dots,c_n+d_n$ is not a solution because $b_j\neq 0$ for the equation j, we have the following contradiction.
\[a_{j1}(c_1+d_1)+\dots+a_{jn}(c_1+d_1)=(a_{j1}c_1+\dots+a_{jn}c_n)+(a_{j1}d_1+\dots+a_{jn}d_n)\]
\[=b_j+b_j=2\cdot b_j\neq b_j\]
Because $b_j\neq 0$, then $2b_j\neq b_j$. Ad absurdum.
\end{proof}
\begin{problem}[1.5.4]
Determine which of the following are subspaces of $C[a,b]$. Here $a,b,c\in\R$ are fixed and $a<c<b$.
\begin{enumerate}[label=(\alph*)]
    \item $V=\{f\in C[a,b]\ |\ f(c)=0\}$
    \item $V=\{f\in C[a,b]\ |\ f(c)=1\}$
    \item $V=\{f\in D[a,b]\ |\ f'(c)=0\}$
    \item $V=\{f\in D[a,b]\ |\ f' \text{ is constant}\}$
\end{enumerate}
\end{problem}
\begin{enumerate}[label=(\alph*)]
    \item
    \begin{proof}\ \\
    Suppose we did have a 1 element, when $1\cdot f$, where $f\in V$ would be the same $f$, this would mean for all input values, $1\cdot f$ would map to the same values of $f$. This would mean for all $x\in \R$, $1(x)=1$, which means there would be no where $1(c)=0$. Thus $V$ is not a field, because it fails to have a 1 element, ad absurdum.\\
    Side note, what if we had a function $1(x)=\left \{\begin{array}{c|c}
         1& x\neq c  \\
         0 & x=c
    \end{array}\right \}$
    I think this can be the one identity.
    \end{proof}
    \item
    \begin{proof}\ \\
    The function $f(x)=1$ would be contained in $V$. However, $(f+f)(x)$ would not, because $(f+f)(x)=f(x)+f(x)=2$ for all $x$.\\
    Thus $V$ is not a field, because it fails closure over addition.
    \end{proof}
    \item
    I don't believe this is a field, still thinking about it.
    \item
    \begin{proof}\ \\
    It is assumed, these are derived, or integrated with respect to $x$, however this is just a placeholder, it can be any singular variable differentiation.\\
    We have closure over addition. Let $f,g\in \F$, and because $f'$ and $g'$ are constants, then we know that $f'(x)=c$ for all x, and $g'(x)=d$ for all x. Thus $f=\int f'\ dx=\int c\ dx= cx+C_1$, and $g=\int g'\ dx=\int d\ dx= dx+C_2$. And $(f+g)(x)=cx+dx+C_1+C_2=(c+d)(x)+C$. Thus $\frac{d}{dx}(c+d)=(c+d)$ for all x. And thus closure is maintained.\\
    However, closure over multiplication does not exist. Suppose it did. $f=\int f'\ dx$ and $g=\int g'\ dx$. By the same argument as before, $f=cx+C_1$ and $g=dx+C_2$, thus $(f\cdot g)=cdx^2+cC_2x+dC_1x+C_1C_2$, which means $(f+g)'=2cdx+dC_1+cC_2$, and it is not constant(when $c\neq 0$ and $d\neq 0$). Thus $V$ is not a field because it fails closure over multiplication.
    \end{proof}
\end{enumerate}
\begin{problem}
Let $V$ be a vector space, and suppose that $U$ and $W$ are both subspaces of $V$. Show that
\[U\cap W := \{v\ |\ v\in U \land v\in W\}\] is also a subspace.
\end{problem}
\begin{proof}\
\begin{enumerate}[label=(\alph*)]
\item If $U$ and $W$ are subspaces of $V$, then they both contain $\vec{0}$, and so $\vec{0}\in U \cap W$.
\item By the same reasoning, $\vec{1}$ is contained in $U\cap W$. This is because $\vec{1}\in V$
\item Say $\vec{u}\in U\cap W$, then $\vec{u}\in U$ and $\vec{u}\in W$, then because of closure over scalar multiplication for $U$ and $W$, we know for $c\in \F$, $c\vec{u}\in U$ and $c\vec{u}\in W$, so we have closure over scalar multiplication for $U\cap W$, because $c\vec{u}\in U\cap W$.
\item Say $\vec{u},\vec{v}\in U \cap W$, then $\vec{u},\vec{v}\in U$ and $\vec{u},\vec{v}\in W$, and because $U$ and $W$ are vector spaces, we know they have closure over vector addition. So $\vec{u}+\vec{v}\in U$ and $\vec{u}+\vec{v}\in W$, so $\vec{u}+\vec{v}\in U\cap W$, thus we have closure over vector addition for $U\cap W$.
\end{enumerate}
Thus $U\cap W$ is a subspace of $V$, because $U\subseteq V$ and $W \subseteq V$, thus $U\cap W \subseteq V$ and $U\cap W$ is a vector space.
\end{proof}
\begin{problem}
Problem 1.5.9 is not so terribly important in its own right, but becomes much
more important in the case that A is actually a vector space over the field $\F$. We
have to refine the definition just a bit, however. If $V$ is a vector space over a field $\F$,
we call $V^*$ = $\{T: V \to \F\ |\ T \text{ is linear}\}$ the dual space of $V$. Addition and scalar
multiplication of elements of $V^*$ is just like in 1.5.9, but one also has to check
that sums and scalar multiples of linear functions are in fact still linear. Worth
doing, but does not to be written up nicely – you may assume that $V^*$ is a vector space. Can you give some geometrically interesting examples of elements of $(\R^2)*$?
Since $V*$ is a vector space, its dual $V^{**}$ = $(V^*)^*$ is also a vector space. To get some idea what sort of critters inhabit it, let’s fix a vector $\vec{v} \in V$ and define a function $ev_v := V^* \to \F$ by $ev_v(T) = T(v)$. Show that $ev_v$ is indeed in $V^{**}$. Then define $\sigma: V \to V^{**}$ by $\sigma(v) = ev_v$. Show that $\sigma$ is a linear transformation.
\end{problem}
\end{document}

