\documentclass[12pt]{article}   % The [12pt] is optional for senior eyes; the default is 10 pt

\setlength{\parindent}{0pt}
\pdfpagewidth 8.5in
\pdfpageheight 11in

\usepackage{amsmath}  % It's generally a good idea to include this if you're going to use any "unusual" symbols. 
\usepackage{amssymb}  % More out of the way math symbols such as \nmid for does not divide
\usepackage{amsfonts}  % I am using this to get the "Blackboard Bold" symbols. 

\usepackage{enumerate}  % The standard enumerate environment numbers the items consecutively. This package enables me to use letters instead. It's optional.
\usepackage{amsthm}

\usepackage{mdwlist}
\usepackage[margin=1.2in]{geometry}

% The following are definitions for shortcuts for pieces of notation that I use a lot.
\newcommand{\N}{\mathbb{N}} % the natural numbers
\newcommand{\Z}{\mathbb{Z}} % the integers
\newcommand{\C}{\mathbb{C}} % the complex numbers 
\newcommand{\R}{\mathbb{R}} % the real numbers
\newcommand{\Q}{\mathbb{Q}} % the rational numbers
\newcommand{\F}{\mathbb{F}} % the Field
\renewcommand{\a}{\alpha}
\renewcommand{\b}{\beta}
\newcommand{\lgr}{\lambda}
\newcommand{\from}{\!\!:} % squeeze the colon closer to the name of the domain of a function
\newcommand{\iso}{\cong} % isomorphism
\newcommand{\rmth}{^{\text{th} } } % roman th as superscript
\newcommand{\inv}{^{-1}} % inverse (exponent -1)
\newcommand{\units}{^{\times}} % groups of units as in E^x

% operator names (set in Roman to contrast with math letters 
\newcommand{\Aut}{\operatorname{Aut}}
\newcommand{\sgn}{\operatorname{sgn}}
\newcommand{\Sym}{\operatorname{Symm}}
\newcommand{\lcm}{\operatorname{lcm}}
\newcommand{\rad}{\operatorname{rad}}
\newcommand{\Gal}{\operatorname{Gal}}

% lower case boldface letters
\newcommand{\uu}{\mathbf u}
\newcommand{\vv}{\mathbf v}
\newcommand{\ww}{\mathbf w}
\newcommand{\ii}{\mathbf i}
\newcommand{\jj}{\mathbf j}
\newcommand{\kk}{\mathbf k}


\newtheorem{theorem}{Problem}
\newtheorem{corollary}{Corollary}[theorem]
\newtheorem{lemma}[theorem]{Lemma}

\newcommand{\th}{^{\text{th} } } % roman th as superscript
\newcommand{\st}{^{\text{st} } } % roman st as superscript
\newcommand{\nd}{^{\text{nd} } } % roman nd as superscript



\title{MATH544 \\ Dr. Miller \\ HW 2}
\author{Justin Baum}

\begin{document}

\maketitle
\begin{theorem}[1.3.6]
Let $\Vec{v_1}$, $\Vec{v_2}$, $\Vec{v_3} \in \R^3$, and suppose that the linear system
\[x\Vec{v_1}+y\Vec{v_2}+z\Vec{v_3}=\Vec{0}\]
has infinitely many solutions. show that $\Vec{v_1}$, $\Vec{v_2}$, $\Vec{v_3} \in \R^3$ lie in a  plane containing the origin in $\R^3$.
\end{theorem}
\begin{proof}
\[\Vec{v_1}=\begin{bmatrix}x_1\\y_1\\z_1\end{bmatrix};\ \ 
\Vec{v_2}=\begin{bmatrix}x_2\\y_2\\z_2\end{bmatrix};\ \ \Vec{v_3}=\begin{bmatrix}x_3\\y_3\\z_3\end{bmatrix} \]
Then
\[x\cdot x_1+y\cdot x_2+z\cdot x_3=0\]
\[x\cdot y_1+y\cdot y_2+z\cdot y_3=0\]
\[x\cdot z_1+y\cdot z_2+z\cdot z_3=0\]
The trivial solution is $x=y=z=0$, but that only gives us one solution. However we can show that one vector is some linear combination of the other 2.

\[x\cdot x_1+y\cdot x_2= -z\cdot x_3\]
\[x\cdot y_1+y\cdot y_2= -z\cdot y_3\]
\[x\cdot z_1+y\cdot z_2= -z \cdot z_3\]
If $x\neq 0\land y \neq 0 \land z\neq 0$, then this can be represented in vector form again.
\[x\begin{bmatrix}x_1\\y_1\\z_1\end{bmatrix}+
y\begin{bmatrix}x_2\\y_2\\z_2\end{bmatrix}=
-z\begin{bmatrix}x_3\\y_3\\z_3\end{bmatrix}\]
Thus one of the vectors is collinear to the addition of the other two. And so $<\Vec{v_1},\Vec{v_2},\Vec{v_3}>=<\Vec{v_1},\Vec{v_2}>$. Because $<\Vec{v_1},\Vec{v_2}>$ includes the $\Vec{0}$(from the given), and all three vectors, because it is the span of 2 and the third is a linear combination of the other two, covered by the definition of span. If this forms a plane, then it will include all 4 vectors. If they both are collinear it will form a line, and thus there are an infinite number of planes that will contain all 4 points. And if it forms a point, then that point must be \Vec{0}, and again there are an infinite number of planes from that point\\\\
If $x=0$,
\[0\begin{bmatrix}x_1\\y_1\\z_1\end{bmatrix}+
y\begin{bmatrix}x_2\\y_2\\z_2\end{bmatrix}=
-z\begin{bmatrix}x_3\\y_3\\z_3\end{bmatrix}\implies y\begin{bmatrix}x_2\\y_2\\z_2\end{bmatrix}=
-z\begin{bmatrix}x_3\\y_3\\z_3\end{bmatrix}\]
Thus we get two collinear vectors, and the same argument, this can be extended to $y=0$ and $z=0$.\\
Suppose we can have 2 solutions that are 0. Then we get 
\[0\begin{bmatrix}x_1\\y_1\\z_1\end{bmatrix}+
0\begin{bmatrix}x_2\\y_2\\z_2\end{bmatrix}=
-z\begin{bmatrix}x_3\\y_3\\z_3\end{bmatrix}\]
Thus \[\begin{bmatrix}0\\0\\0\end{bmatrix}=
-z\begin{bmatrix}x_3\\y_3\\z_3\end{bmatrix}\]
And so we have the null vector as one of the vectors in the linear combination, and we can just draw a plane through the 3 points(or line if they are collinear, and a point if they are all the null vector.\\
If 3 solutions are 0, this is trivially true.
\end{proof}

\end{document}