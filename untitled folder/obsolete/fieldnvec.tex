\begin{defn}
A field is a set $\F$ with operations of addition $(+)$ and multiplication $(\cdot)$ and distinct elements: $0$, $1$ so that
\begin{enumerate}
    \item[(C1)] Closure under addition: if $a,b\in\F$ then $a+b\in\F$.
    \item[(C2)] Closure under multiplication: if $a,b\in\F$ then $a\cdot b\in \F$.
    \item[(A1)] Addition is commutative, $a+b=b+a$ for all $a,b\in\F$.
    \item[(A2)] Addition is associative, $(a+b)+c=a+(b+c)$, for all $a,b,c\in\F$.
    \item[(A3)] There exists a $0$ identity, $a+0=0+a=a$, for all $a\in\F$.
    \item[(A4)] For each $a\in\F$ there exists an element $b$ so that $a+b=b+a=0$.\\
    This element is indeed unique, and we write it as $b=-a$.
    \item[(M1)] Multiplication is commutative, $ab=ba$ for all $a,b\in\F$.
    \item[(M2)] Multiplication is associative, $a(bc)=(ab)c$ for all $a,b,c\in\F$.
    \item[(M3)] There exists a $1$ identity, $1a=a$, for all $a\in\F$.
    \item[(M4)] For each $a\in\F\setminus \{0\}$ there exists an element so that $ab=ba=1$.\\
    This element is indeed unique, and we write it as $b=a^{-1}$.
    \item[(D)] For all $a,b,c\in\F$, multiplication is distributive, $a(b+c)=ab+ac$.
 \end{enumerate}
\end{defn}
\begin{defn}
A set $V$ with binary operations $(+)$ and scalar multiplication is called a vector space over a field $\F$ if:
\begin{enumerate}
    \item[(C1)] The set is closed under addition, for all $\vu,\vv\in V$, $\vu + \vv \in V$.
    \item[(C2)] The set is closed under scalar multiplication, for all $c\in\F$ and $\vu\in V$, $c\vu \in V$.
    \item[(A1)] Vector addition is commutative, $\vu+\vv = \vv + \vu$, for all $\vu,\vv \in V$.
    \item[(A2)] Vector addition is commutative, $(\vu+\vv)+\vw=\vu+(\vv+\vw)$, for all $\vu,\vv,\vw\in V$.
    \item[(A3)] There is an element $\Vec{0}\in V$, so that $\vec{0}+\vu=\vu+\vec{0}$ for all $\vu\in V$.
    \item[(A4)] For every $\vu\in V$ there exists a vector $\vv\in V$ so that $\vu+\vv=\vec{0}$.\\
    We write $\vv=-\vu$.
    \item[(S1)] Scalar multiplication is associative, $s\cdot(t\cdot\vu)=(s\cdot t)\cdot\vu$ for all $s,t\in \F$ and $\vu\in V$.
    \item[(S2)] Recall that we have $1\in\F$, then $1\cdot \vu=\vu$ for all $\vu\in V$.
    \item[(D1)] $c(\vu+\vv)=c\vu+c\vv$ for all $c\in \F$, $\vu,\vv\in V$.
    \item[(D2)] $(c+d)\vu=c\vu+d\vu$ for all $c\in\F$, $\vu\in V$.
\end{enumerate}
\end{defn}