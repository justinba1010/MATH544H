\documentclass[12pt]{article}   % The [12pt] is optional for senior eyes; the default is 10 pt

\setlength{\parindent}{0pt}
\pdfpagewidth 8.5in
\pdfpageheight 11in

\usepackage{amsmath}  % It's generally a good idea to include this if you're going to use any "unusual" symbols. 
\usepackage{amssymb}  % More out of the way math symbols such as \nmid for does not divide
\usepackage{amsfonts}  % I am using this to get the "Blackboard Bold" symbols. 

\usepackage{enumerate}  % The standard enumerate environment numbers the items consecutively. This package enables me to use letters instead. It's optional.
\usepackage{amsthm}

\usepackage{mdwlist}
\usepackage[margin=1.2in]{geometry}

% The following are definitions for shortcuts for pieces of notation that I use a lot.
\newcommand{\N}{\mathbb{N}} % the natural numbers
\newcommand{\Z}{\mathbb{Z}} % the integers
\newcommand{\C}{\mathbb{C}} % the complex numbers 
\newcommand{\R}{\mathbb{R}} % the real numbers
\newcommand{\Q}{\mathbb{Q}} % the rational numbers
\newcommand{\F}{\mathbb{F}} % the Field
\renewcommand{\a}{\alpha}
\renewcommand{\b}{\beta}
\newcommand{\lgr}{\lambda}
\newcommand{\from}{\!\!:} % squeeze the colon closer to the name of the domain of a function
\newcommand{\iso}{\cong} % isomorphism
\newcommand{\rmth}{^{\text{th} } } % roman th as superscript
\newcommand{\inv}{^{-1}} % inverse (exponent -1)
\newcommand{\units}{^{\times}} % groups of units as in E^x

% operator names (set in Roman to contrast with math letters 
\newcommand{\Aut}{\operatorname{Aut}}
\newcommand{\sgn}{\operatorname{sgn}}
\newcommand{\Sym}{\operatorname{Symm}}
\newcommand{\lcm}{\operatorname{lcm}}
\newcommand{\rad}{\operatorname{rad}}
\newcommand{\Gal}{\operatorname{Gal}}

% lower case boldface letters
\newcommand{\uu}{\mathbf u}
\newcommand{\vv}{\mathbf v}
\newcommand{\ww}{\mathbf w}
\newcommand{\ii}{\mathbf i}
\newcommand{\jj}{\mathbf j}
\newcommand{\kk}{\mathbf k}


\newtheorem{theorem}{Problem}
\newtheorem{corollary}{Corollary}[theorem]
\newtheorem{lemma}[theorem]{Lemma}

\newcommand{\th}{^{\text{th} } } % roman th as superscript
\newcommand{\st}{^{\text{st} } } % roman st as superscript
\newcommand{\nd}{^{\text{nd} } } % roman nd as superscript



\title{MATH544 \\ Dr. Miller \\ HW 2}
\author{Justin Baum}

\begin{document}

\maketitle
\begin{theorem}[1.4.10]
A linear system over a field $\F$ is called upper triangular if the coefficient $a_{ij} = 0$ whenever $i > j$.\\
Show that if $a_{ii} \neq 0$ then the system is consistent with a unique solution.
\end{theorem}
\begin{proof}
Base Case: $n=1$.\\
\[a_{11}x_1=b1\]
Trivially we can calculate
\[x_1=\frac{b_1}{a_{11}}\]
Inductive Hypothesis:
Given an (n+1) x (n+1) upper triangular system, and $a_{ii} \neq 0$.\\
If we remove the bottom n rows, and make that into an n x n system with $x_1,...,x_n$, then $E_{n+1}$ will be the (n+1)\textsuperscript{th} linear equation with (n+1) terms.

\[a_{11}x_1+...+a_{1n}x_n=b_1\]
\[\vdots\]
\[a_{nn}x_n = b_n\]
Let $j = n+1$
\[a_{j1}x_1+...+a_{jn}x_{n}+a_{jj}x_{j}=b_{j}\] Due to the base case, the n x n system has a single unique solution, $x_1,\dots,x_n$. Because so,
\[x_{j}=\frac{1}{a_{jj}}\left ( - (a_{j1}x_1+...+a_{jn}x_n) + b_{j}\right )\]
Thus an (n+1) x (n+1) linear system has a unique solution as well.
\end{proof}

\end{document}